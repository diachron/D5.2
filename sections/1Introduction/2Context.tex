%!TEX root = ../../diachron-D5_2.tex

\subsection{Context of this Document}
\label{sec:ContextDoc} 

\todo{CL: my first shot; improvised w/o looking into previous deliverables; need to verify against them - JD: I think this is good!}The overall architecture of Diachron, as outlined in Deliverable 6.1~\cite[section~5]{diachron-d6.1}, comprises i) a platform layer of modules that access and operate on diachronic data, ii) an integration layer that makes these modules accessible as web services, as well as iii) pilot applications that solve domain-specific problems and make use of the services to handle data specific to these problems.  An initial specification of the interfaces of these services – web services with a RESTful API and JSON input/output, to be exact – has also been provided in Deliverable 6.1.  Here, we describe the \emph{implementation} of the \emph{quality assessment module} and the \emph{quality assessment service} and explain, in a generic, domain-independent way, how to use it to assess the quality of data.

The concrete quality metrics that we have so far implemented within our quality assessment framework, and which the service can compute for any given dataset, follow the definitions from Deliverable 5.1~\todo{CL: add to bibtex}\cite{diachron-d5.1}.
We \todo{CL@JD: This is a bit fuzzy, but we didn't implement all of them, did we? - JD: nope.. see comment in previous section}prioritised the implementation of those metrics that Deliverable 5.1 identified relevant to Diachron~\cite[section~4]{diachron-d5.1}.

\todo{CL@JD: We are completely ignoring the tool survey from \cite[section~5]{diachron-d5.1}.  Should we say something that we didn't find any of the existing tools suitable because …? - JD: I say it later i think}

% How does this work relate to WP5 in general and the whole project
% add the image of the whole diachron architecture (6.1)
% ? and highlight the part(s) covered by this deliverable

%%% Local Variables: 
%%% mode: latex
%%% TeX-master: "../../diachron-D5_2"
%%% End: 
