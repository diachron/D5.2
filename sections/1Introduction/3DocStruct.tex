%!TEX root = ../../diachron-D5_2.tex

\subsection{Document Structure}
\label{sec:DocumentStructure} 

Section~\ref{sec:DQF} defines a generic, extensible data quality framework, which is not specific to any Diachron application domain.  The framework allows for implementing arbitrary concrete quality metrics; any such metric is implemented partly in Java and partly as an extension of the data quality (daQ) core vocabulary.  The input dataset, currently assumed to be represented in RDF, is fed into all quality metric implementations of the user's choice; the output is represented as a set of RDF metadata in terms of daQ and its concrete, metric-specific extensions, so that it can be stored and reused flexibly.  This core quality assessment step is accessible through a web service interface.  Supported uses of a dataset's quality metadata include visual quality analysis (over multiple revisions of a dataset if desired) as well as \todo{CL@JD: Interestingly visual analysis is covered in section 2 as a part of the ``framework''; filtering and ranking is not.  Why this difference?}filtering and ranking datasets by their quality.  

Section~\ref{sec:Libraries} lists the libraries that served as technical prerequisites for implementing the quality assessment service.

Section~\ref{sec:RankingService} describes in detail the general processes of data quality assessment and quality-based ranking, the design of a self-contained user interface giving access to the quality assessment service (optionally also the cleaning service described in Deliverables 3.1~\cite{diachron-d3.1} and 3.2~\cite{diachron-d3.2}, as well as all concrete metrics that we have implemented so far, with a special focus on those required for the Diachron pilot applications.

\todo{CL@Hanzo: also introduce section~\ref{sec:CrawlingService} once you have written it}

%%% Local Variables: 
%%% mode: latex
%%% TeX-master: "../../diachron-D5_2"
%%% End: 
