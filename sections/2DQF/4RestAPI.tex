%!TEX root = ../../diachron-D5_2.tex

\subsection{Design of the RESTful Quality API}
\label{sec:RestAPI} 
The RESTful API design and activity diagrams are explained in Deliverable 6.1~\cite[Section 6.1.7]{diachron-d6.1}.
The only minor change is in the input parameters for the \texttt{\url{/diachron/compute_quality}} API call.
In Deliverable 6.1 we define the following two parameters:
\begin{description}
\item[Dataset –] An instance of a DIACHRON dataset URI;
\item[QualityReportRequired –] A boolean indicating whether a quality report is required.
\end{description}
The input parameter we introduce in this Deliverable is \textbf{MetricsConfiguration}.
This parameter is an object with a list of metrics (cf. Listing~\ref{lst:conf_metric}) in JSON-LD format, identifying the metrics required to be used for the dataset quality assessment.
Listing~\ref{lst:api_format} shows a sample input message format with the newly added parameter \textbf{MetricsConfiguration}\todo{CL@JD: If others implement their own metrics in packages other than de.unibonn.iai.eis... how will this work?  It currently implicitly maps to de.unibonn..., doesn't it? - JD: no idea}.
\lstinputlisting[caption={API Call Input Message Format},label=lst:api_format, language=json]{listings/api_format.json} 
% relate to 6.1

%%% Local Variables: 
%%% mode: latex
%%% TeX-master: "../../diachron-D5_2"
%%% End: 
