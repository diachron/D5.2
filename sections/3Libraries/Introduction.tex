%!TEX root = ../../diachron-D5_2.tex

\section{Tools and Libraries Used}
\label{sec:Libraries} 

In this section we discuss the main tools and libraries used in our solution to help us achieve our goal.

\subsection{OntoWiki}
\label{sec:OntoWiki}
OntoWiki\footnote{http://ontowiki.net/} is a tool providing support for agile, distributed knowledge engineering scenarios.
Based on semantic technologies, OntoWiki provides an easy to use control management system (CMS) that allows users to manage the knowledge base (RDF data) underlying the application.
OntoWiki is part of the LOD2 Stack, licensed under GPL and is open source.
The Quality Framework is currently based on top of OntoWiki.

\subsection{CubeViz}
\label{sec:CubeViz}
CubeViz\footnote{http://cubeviz.aksw.org} is an RDF DataCube browser and also an extension to OntoWiki.
This extension allows users to visually represent statistical data represented in RDF, specifically data which is modelled by the RDF DataCube vocabulary.
The Dataset Quality Vocabulary (daQ) is defined to use the mentioned RDF statistical vocabulary, thus CubeViz was a suitable extension to use to visualise statistical results about quality metadata.


\subsection{Apache Jena}
\label{sec:Jena}
Apache Jena\footnote{https://jena.apache.org} is an open source Java framework for building Semantic Web and Linked Data applications.
The framework is composed of different APIs interacting together to process RDF data. If you are new here, you might want to get started by following one of the tutorials.
Apache Jena is licensed under the Apache License, Version 2.0. 
This framework is the underlying technology used for the ``Quality Core Framework''.

% mention all libraries used and their licences
