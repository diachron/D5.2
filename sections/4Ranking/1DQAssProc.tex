%!TEX root = ../../diachron-D5_2.tex

\subsection{Data Quality Assessment Process}
\label{sec:DQAssessment} 
In Deliverable 6.1 (Section 6.1.7.3), we described the assessment process in an activity diagram.
Since we need to cater for datasets modelled on the DIACHRON Data Model (cf. D1.3), the Stream Processor described in Section~\ref{sec:streamProcessor} requires some adaptation.
The DIACHRON data model is used as an underlying schema to store and query data from various heterogeneous sources using one standardised representation.
The schema features a number of key aspects that allow data to be traceable (indicating the provenance), reproducible, and overall enable the DIACHRON services (cf. D1.3).

Due to the nature of the heterogenous data that might originate from different models (e.g. triple-based, tuple-based, multidimensional etc.), the DIACHRON model is highly based on reified RDF.
This enables the description of resources as RDF statements (expanding a triple to a reification quad\footnote{\url{http://www.w3.org/TR/2004/REC-rdf-primer-20040210/#reification}}), recording further information such as provenance, attributes, relations and also changes.
Therefore, if we take for example the triple\footnote{This example does not follow the DIACHRON data model, but rather to illustrate a simple example of reification}:
\begin{lstlisting}[language=RDF]
ex:resourceSubject ex:property "value" .
\end{lstlisting}
it will be represented as a set of \texttt{rdf:Statement} triples:
\begin{lstlisting}[language=RDF]
ex:refiedResource rdf:type rdf:Statement .
ex:refiedResource rdf:subject ex:resourceSubject .
ex:refiedResource rdf:predicate ex:property .
ex:refiedResource rdf:object "value" .
\end{lstlisting}

In Section~\ref{sec:streamProcessor} we described how the \emph{Stream Processor} sequentially passes quads to the metrics in the \emph{Quality Assessment Layer}.
Although data based on the DIACHRON data model would be sufficient for the \emph{Stream Processor}, metric assessment will not give the desired results.
The main problem is that the most metrics would not just require the \emph{object} value, but would also require the \emph{subject} and the \emph{predicate}.
Therefore, DIACHRON resources have to be de-reified into RDF triple prior to them being passed to the metric.



% describe the process (similar to the activity diagram in 6.1). Mention the initialisation of metrics as required for each use case, and how the streaming process is done