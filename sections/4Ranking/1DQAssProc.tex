%!TEX root = ../../diachron-D5_2.tex

\subsection{Data Quality Assessment Process}
\label{sec:DQAssessment} 
In Deliverable 6.1 (Section 6.1.7.3), we described the assessment process in an activity diagram.
Since we need to cater for datasets modelled on the DIACHRON Data Model (cf. D1.3), the Stream Processor described in Section~\ref{sec:StreamProcessor} requires some adaptation.
The DIACHRON data model is used as an underlying schema to store and query data from various heterogeneous sources using one standardised representation.
The schema features a number of key aspects that enable DIACHRON services, in particularly making data traceable (indicating the provenance) and reproducible (cf. D1.3).

Due to the heterogenous nature of data from different models (e.g. triple-based, tuple-based, multidimensional etc.), the DIACHRON model is highly based on reified RDF.
This enables the description of resources as RDF statements (expanding a triple to a reification quad\footnote{\url{http://www.w3.org/TR/2004/REC-rdf-primer-20040210/#reification}}), recording further information such as provenance, attributes, relations and also changes.
For example, if consider the triple\footnote{This example does not follow the DIACHRON data model, but rather to illustrate a simple example of reification}:
\begin{lstlisting}[language=N3]
ex:resourceSubject ex:property "value" .
\end{lstlisting}
it will be represented as a set of \texttt{rdf:Statement} triples:
\begin{lstlisting}[language=N3]
ex:reifiedResource rdf:type rdf:Statement .
ex:reifiedResource rdf:subject ex:resourceSubject .
ex:reifiedResource rdf:predicate ex:property .
ex:reifiedResource rdf:object "value" .
\end{lstlisting}

In Section~\ref{sec:StreamProcessor} we described how the \emph{Stream Processor} sequentially passes quads to the metrics in the \emph{Quality Assessment Layer}.
Although data based on the DIACHRON data model would be sufficient for the \emph{Stream Processor}, metric assessment will not give the desired results.
The main problem is that most metrics would not just require the \emph{object} value, but would also require the \emph{subject} and the \emph{predicate}.
Therefore, DIACHRON resources have to be de-reified into RDF triple prior to them being passed to the metric.

In order to support the DIACHRON data model, datasets would require preprocessing to be de-reified to a flat RDF structure.
This can be done in two ways: (1) Constructing a new dataset using SPARQL CONSTRUCT; (2) Parsing a dump of the dataset.
These are explained further below.

\paragraph{Constructing de-reified RDF triples}~\\
This is the preferred method of flattening down RDF triples from the DIACHRON data model.
For this, a SPARQL endpoint is desirable so that we could query the model directly and do not require loading the dataset into memory for the construction of the model.
Listing~\ref{lst:dereifiedRDF} illustrates the SPARQL query used to de-reify a dataset.
The place holder \texttt{::datasetUri::} will be replaced by the dataset URI provided.
\begin{lstlisting}[language=sparql,caption="SPARQL Query to de-reify the DIACHRON data model into flat RDF",label=lst:dereifiedRDF]
CONSTRUCT {
	?s ?p ?o
}
WHERE{
	::datasetUri:: diachron:hasRecordSet ?rs .
	GRAPH ?rs {?rs diachron:hasRecord [diachron:subject ?s ;
	diachron:hasAttribute [diachron:Property ?p ; diachron:object ?o] ]
	}
}
\end{lstlisting}
Once the graph is constructed, it is passed to the stream processor which in turn starts the assessment.

\paragraph{Parsing a dump of the dataset}~\\
In the unlikely event that a SPARQL endpoint is not available, a data wrapper is provided with the stream processor module.
This data wrapper parses the RDF statements and creates triples on the fly, which are then passed to the stream processor to sequentially deliver into the \emph{Quality Assessment Layer}.

%%% Local Variables: 
%%% mode: latex
%%% TeX-master: "../../diachron-D5_2"
%%% End: 
