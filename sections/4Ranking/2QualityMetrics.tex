%!TEX root = ../../diachron-D5_2.tex

\subsection{Data Quality Metrics}
\label{sec:DQMetrics} 

% describe what metrics have been implemented - separate in categories.
% Add more categories which metrics were implemented in, even if they are not important for ebi use case
% point to D5.1 for full details of the metrics


%Introduction
In this subsection we describe the metrics implemented as at the deliverable's deadline which are included in the first prototype of this service.
For this purpose, those metrics which are generally required by DIACHRON (\textbf{GEN}) and those specific by the pilot partners (Data Publica – \textbf{DP};  Scientific Pilot – \textbf{EBI}; cf. Deliverable 5.1~\cite[section 4]{diachron-d5.1}) were analysed and prioritised accordingly, also keeping in mind metrics required by the cleaning application discussed in Deliverable 3.2~\cite{diachron-d3.2}.
Some pilot partners required custom metrics (such as checking for \todo{CL@JD: I don't understand.  Do you mean a \emph{specific} property?}an additional property) to be implemented.
These will be identified with a $\star$.
Each implemented metric is expected to conform to the interface described in Section~\ref{sec:processingUnit}.

%!TEX root = ../../../diachron-D5_2.tex

\subsubsection{Accessibility Category}
\label{sec:Accessibility} 

\paragraph{Availability Dimension} % dimension name 

\paragraph{Dereferenceability Metric} ~\\ %metric name
% background for the metric
HTTP URIs should be dereferencable, i.e. HTTP clients can retrieve the resources identified by the URI.
A typical web URI resource would return a \texttt{200 OK} code indicating that a request is successful and \texttt{4xx} or \texttt{5xx} if the request is unsuccessful. 
In Linked Data, a successful request should return a document (RDF) containing the description (triples) of the requested resource.
In Linked Data, there are two possible ways which allow publishers make URIs dereferencable.
These are the \texttt{303} URIs and the \texttt{hash} URIs\footnote{http://www.w3.org/TR/cooluris/}.
Yang et. al~\cite{Yang2011} describes a mechanism to identify the dereferenceability process of linked data resource.
% short description
\begin{mdframed}[style=metricdefinition]
\emph{Calculates the number of valid redirects (303) or hashed links according to LOD Principles.}
\end{mdframed}

% pseudocode
This metric (listing~\ref{let:deref}) will count the number of valid dereferenceable URI resources found in the subject (?s) and object (?o) position of a triple. The \texttt{isDereferenceable(resource)} method uses the rules defined in~\cite{Yang2011}.
\begin{algorithm}
\caption{Dereferenceablity Algorithm}\label{lst:deref}
\begin{algorithmic}[1]
\Procedure{init}{}
\State totalTriples = 0 ;
\State deref = 0 ;
\EndProcedure

\Procedure{Dereference}{$\langle?s,?p,?o,?g\rangle$}
\If {(isURI(?s)) \&\& (isDereferenceable(?s))} deref++ ; \EndIf

\If {(isURI(?o)) \&\& (isDereferenceable(?o))} deref++ ; \EndIf
\State totalTriples++;
\EndProcedure
\end{algorithmic}
\end{algorithm}
% metric value, range and rating
The metric will return a ratio of the number of dereferenced URIs (deref) against the total number of triples in a dataset (totalTriples). The expected range is [0..1], where 0 is the worst rating and 1 is the best rating.


%%%%%%% Bibliography
% Yang2011 : https://dl.dropboxusercontent.com/u/4138729/paper/dereference_iswc2011.pdf
%!TEX root = ../../../diachron-D5_2.tex

\subsubsection{Intrinsic Category}

The intrinsic category metrics are independent of the user's context.
They reflect whether information presented in data correctly represent the real world and whether information is logically consistent itself.
\label{sec:Intrinsic} 


\subsubsection{Accuracy} % dimension name 

\paragraph{Malformed Datatype Literals} ~\\ %metric name
% background for the metric
% short description

Literals are nodes in an RDF graph, used to identify values such as numbers and dates.
A plain literal is a string combined with an optional language tag.
A typed literal comprises a string (the lexical form of the literal) and a datatype (identified by a URI) which is supposed to denote a mapping from lexical forms to some space of values.
The RDF specifies two types of literals: plain and typed. 
In the Turtle syntax typed literals are notated with syntax such as:  \textit{"13"^^xsd:int}.
Malformed datatype literals are quit common and can be fixed by simple changes to the literals.
This Malformed Datatype Literals metric intends to check if the value of a typed literal is valid with regards to the given xsd datatype.

\begin{mdframed}[style=metricdefinition]
\emph{Calculates the ratio of typed literals not valid regarding its given xsd datatype to all literals}
\end{mdframed}

% pseudocode


\begin{algorithm}
\caption{Malformed Datatype Literals Algorithm}\label{lst:deref}
\begin{algorithmic}[1]
\Procedure{init}{}
\State totalLiterals = 0 ;
\State malformedLiterals = 0 ;
\EndProcedure

\Procedure{ComputeMalformedDatatypeLiterals}{$\langle?s,?p,?o,?g\rangle$}
\If {(isLiteral(?o))} totalLiterals++ ;
\If {(isTypedLiteral(?o)) \&\& (!hasValidDatatype(?o))} malformedLiterals++ ; \EndIf
 \EndIf
\State metric=malformedLiterals/totalLiterals;
\EndProcedure
\end{algorithmic}
\end{algorithm}
% metric value, range and rating

\paragraph{Literals Incompartible with Datatype Range} ~\\ 

\paragraph{Defined Ontology Author} ~\\ 
\paragraph{POBO Definition Usage}~\\ 
\paragraph{Synonym Usage}~\\ 
\subsubsection{Consistency}~\\ 
\

\paragraph{Entities As Members of Disjoint Classes}~\\


\paragraph{Homogeneous Datatypes}~\\
 
\paragraph{Misplaced Classes or Properties}
\paragraph{Misused Owl Datatype or Object Properties}
\paragraph{Obsolete Concepts in Ontology}
\paragraph{Ontology Hijacking}
\paragraph{Undefined Classes Metric} ~\\ 
\label{par:undefClass}

Oftentimes a terms which is used in the object position of a triple and is not a literal is not formaly defined as being a class.
'Being defined' means that the term is defined either in some external ontology or at an earlier position in the given dataset. 
Regarding to Hogar ~\cite{hogan2010} to the most used undefined classes belong \texttt{foaf:UserGroup}, \texttt{rss:item}, \texttt{linkedct:link}, \texttt{politico:Term} .
The probability for undefined class in the subject position is very low, because the subject of a quad never references classes or properties in external vocabularies.
Therefore they is no need to analyse the subject for this metric.
For the most LOD data sets is sufficient to check object by the predicate \texttt{rdf:type}. 
In the case when LOD data set defines its own vocabulary the following predicates indicate that the object must be a defined class:  \texttt{rdfs:domain}, \texttt{rdfs:range}, \texttt{rdfs:subClassOf}, \texttt{owl:allValuesFrom}, \texttt{owl:someValuesFrom}, \texttt{owl:equivalentClass}, \texttt{owl:complementOf}, \texttt{owl:onClass}, \texttt{owl:disjointWith}.
The undefined classes problem occurs due to spelling or syntactic mistakes resolvable through minor fixes to the respective ontologies.
The missing classes should be define in corresponding ontology or in a separate namenspace.


\begin{mdframed}[style=metricdefinition]
\emph{Calculates the ratio of undefined classes to all classes in object position in a dataset}
\end{mdframed}


\begin{algorithm}
\caption{Undefined Classes Metric Algorithm}\label{lst:undefCl}
\begin{algorithmic}[1]
\Procedure{init}{}
\State totalClasses = 0 ;
\State undefinedClasses = 0 ;
\EndProcedure

\Procedure{ClassNotDefined}{$\langle?s,?p,?o,?g\rangle$}
\If {(isClassProperty(?p) \&\& (isURI(?o))} ; 

\If {(!isDefined(?o))  undefinedClasses++ }; \EndIf
\EndIf
\State totalClasses++ ++;
\EndProcedure
\end{algorithmic}
\end{algorithm}

\paragraph{Undefined  Properties}  ~\\ 


Similar to the Undefined Classes metric \ref{par:undefClass} the Undefined Properties metric identifies terms in the predicate position that are used without any formal definition.
 Hogan  ~\cite{hogan2010} identified the following properties that are often used whithout being defined:  \textit{foaf:image}, \textit{cycann:label} , \textit{foaf:tagLine}.
The following list of predicates indicate that the object of the quad must be a defined property:\textit{rdfs:subPropertyOf}, \textit{owl:onProperty}, \textit{owl:assertionProperty}, \textit{owl:equivalentProperty}, \textit{owl:propertyDisjointWith}.

\begin{mdframed}[style=metricdefinition]
\emph{Calculates the ratio of undefined properties to all properties in the given data set}
\end{mdframed}

\begin{algorithm}
\caption{Undefined Properties Algorithm}\label{lst:undefCl}
\begin{algorithmic}[1]
\Procedure{init}{}
\State totalProperties = 0 ;
\State undefinedProperties = 0 ;
\EndProcedure

\Procedure{PropertyNotDefined}{$\langle?s,?p,?o,?g\rangle$}
\If {(isURI(?p) totalProperties++} ; 
\If {(!isDefined(?p)) udefinedProperties++ }; \EndIf

\If {(isFromList(?p)) \&\&  !isDefined(?o) udefinedProperties++ }; \EndIf
\State totalProperties ++;
\EndIf

\EndProcedure
\end{algorithmic}
\end{algorithm}


\subsubsection{Conciseness}

\paragraph{Duplicate Instance}

\paragraph{Extensional Conciseness}

\paragraph{Ontology Versioning Conciseness}





\subsubsection{Representational Category}
\label{sec:Representational} 

Representational dimensions reflect the quality aspects like conciseness, consistency and interpretability of information.

\subsubsection{Representational Conciseness}

\paragraph{Short URIs} ~\\
% background for the metric
URIs play a key role in how information is represented in Linked Data resources, as they are used to name the entities being described. Therefore, having compact, well formatted URIs has a positive effect in the clearness and conciseness of data. 
As suggested by \cite{Hogan2012}, data providers that locally mint (on average) shorter URIs are deemed as being more compliant with Linked Data best practices.

% short description
\begin{mdframed}[style=metricdefinition]
\emph{Detects whether, in average, short URIs are being used, which suggests that information is compactly represented and that readability is favored.}
\end{mdframed}

% pseudocode
Implementation details regarding these metric are provided in algorithm~\ref{alg:shortURIs}. All URIs identifying instances, that are defined locally, are considered by the metric. The calculation is performed as the average of the lengths of the URIs corresponding to the subjects of all instance declarations (i.e. statements using the \textit{rdf:type} predicate).
\begin{algorithm}
\caption{Short URIs Algorithm} \label{alg:shortURIs}
\begin{algorithmic}[1]
\Procedure{init}{}
\State accumulatedURIsLength = 0;
\State countLocallyDefURIs = 0;
\EndProcedure
\Procedure{compute}{$\langle?s,?p,?o,?g\rangle$}
\If {equalsURI(?p, rdf:type) \&\& isURI(?s)} 
\State accumulatedURIsLength += lengthOfURI(?s);
\State countLocallyDefURIs++;
\EndIf ~\\
\Return{accumulatedURIsLength / countLocallyDefURIs;}
\EndProcedure
\end{algorithmic}
\end{algorithm}
% metric value, range and rating
The metric will return the average length of all the URIs locally defined in the dataset. The expected range is a real number in the range [0, $+\infty$). Lower values represent better rankings.

% end-of Short URIs Metric

\subsubsection{Understandability}

Understandability is the quality of information that enables users to comprehend its meaning. 
Understandability is an ultimate prerequisite for information consumer. 
The better users understand data, the more effectively they can use it. 
The metrics presented below reflect different aspects of understandability by a numerical value.

\paragraph{Empty Annotation Value}
\label{par:emptyAnnotation}
In some languages, e.g. OWL annotation properties are distinguished.
Annotation properties are predicates that provide informal documentation annotations about ontologies, statements, or IRIs. 
A simple example for annotation property is \textit{rdfs:comment} which is used to provide a comment. 
Unfortunately annotation properties are often used with empty literal values that cause inconsistences in data.
The problem can be solved by the corresponding triples or by replacing empty literals by annotation strings.
The following annotation properties were used in this metric:
\begin{itemize}
\item \textit{skos:altLabel}
\item \textit{skos:hiddenLabel}
\item \textit{skos:prefLabel}
\item \textit{skos:changeNote}
\item \textit{skos:definition}
\item \textit{skos:editorialNote}
\item \textit{skos:example}
\item \textit{skos:historyNote}
\item \textit{skos:note}
\item \textit{skos:scopeNote}
\item \textit{dcterms:description}
\item \textit{dc:description}
\item \textit{rdf:label}
\item \textit{rdf:comment}
\end{itemize}

% background for the metric
The metric Empty Annotation Value identifies triples whose property is an annotation property and whose object is an empty string.
 
% short description
\begin{mdframed}[style=metricdefinition]
\emph{Calculates the ratio of annotations with empty values to all annotations in the data set.}
\end{mdframed}

The algorithm \ref{lst:emptyAnnotationValue} provides detailed description of metric computation. The algorithm first identifyes annotation properties, that are properties from the list above and then checks if the property value is not an empty string.
% pseudocode
\begin{algorithm}
\caption{Empty Annotation Value Algorithm}\label{lst:emptyAnnotationValue}
\begin{algorithmic}[1]
\Procedure{init}{}
\State totalAnnotations = 0 ;
\State emptyAnnotations = 0 ;
\EndProcedure

\Procedure{compute}{$\langle?s,?p,?o,?g\rangle$}
\If {isAnnotation(?p)} totalAnnotations++; \EndIf
\If {isEmpty(?o)} emptyAnnotations++ ; \EndIf
\Return{emptyAnnotations/totalAnnotations}
\EndProcedure
\end{algorithmic}
\end{algorithm}


The metric values vary in the interval [0,1], where  the 0 indicates the best quality.


\paragraph{Whitespace in Annotation}


In contrast to the previous metric \ref{par:emptyAnnotation} which identifies triples with empty annotation value, the Whitespace in Annotation metric deals with the case when the annotation value is available, but contains leading or trailing whitespaces.
This is not a crucial for data understanding, but should be at least reported to the user who wants to improve data quality.
The metric is defined as follows:

% short description
\begin{mdframed}[style=metricdefinition]
\emph{Calculates the ratio of annotations with leading or trailing whitespace to all annotations in the data set.}
\end{mdframed}

An important prerequisite for metric computation is a predefined list of annotation properties  which we presented in the previous metric \ref{par:emptyAnnotation}.
The algorithm \ref{lst:whitespace} provides detailed description of metric computation. The algorithm first identifyes annotation properties, that are properties from the list above and then checks if the property value is not an empty string.


% pseudocode
\begin{algorithm}
\caption{Whitespace in Annotation Algorithm}\label{lst:whitespace}
\begin{algorithmic}[1]
\Procedure{init}{}
\State totalAnnotations = 0 ;
\State whitespaceInAnnotations = 0 ;
\EndProcedure

\Procedure{compute}{$\langle?s,?p,?o,?g\rangle$}
\If {isAnnotation(?p)} totalAnnotations++; \EndIf
\If {containsLeadingORTrailingWhitespace(?o)} whitespaceInAnnotations++ ; \EndIf
\Return{whitespaceInAnnotations/totalAnnotations}
\EndProcedure
\end{algorithmic}
\end{algorithm}

The \textit{isAnnotation()} method  only checks if the property belongs to the list of annotation properties, 

The metric values vary in the interval [0,1], where  the 0 indicates that data set is free of the leading or trailing whitespace in annotations.


\paragraph{Labels Using Capitals}

Similar to the two previous metrics the metric Labels Using Capitals considers a specific kind of properties, namely label properties.
The metrics identifies the triples with label property whose object uses a bad style of capitalisation. 
We define "bad" capitalisation as "camel case" where compound words or phrases  are written such that each next word or abbreviation begins with a capital letter, e.g. \textit{InterestingThing}.
The following widely used label properties are considered by the metric:
\begin{itemize}
\item \textit{skos:altLabel}
\item \textit{skos:hiddenLabel}
\item \textit{skos:prefLabel}
 \item \textit{rdfs:label}
		



\begin{mdframed}[style=metricdefinition]
\emph{Calculates the ratio of labels with "bad capitalisation" to all labels}
\end{mdframed}

The algorithm \ref{lst:badCapitals} first identifyes label properties and using a regular expression checks if the property value is bad capitalized.


% pseudocode
\begin{algorithm}
\caption{Labels Using Capitals Algorithm}\label{lst:badCapitals}
\begin{algorithmic}[1]
\Procedure{init}{}
\State totalLabels = 0 ;
\State badCapitalizedLabels = 0 ;
\EndProcedure

\Procedure{compute}{$\langle?s,?p,?o,?g\rangle$}
\If {isLabel(?p)} totalLabels++; \EndIf
\If {isBadCapitalized(?o)}badCapitalizedLabels++ ; \EndIf
\Return{badCapitalizedLabels/totalLabels}
\EndProcedure
\end{algorithmic}
\end{algorithm}


The metric values vary in the interval [0,1], where  the 0 indicates that data set doesn't contain bad capitalized labels.

\paragraph{Human Readable Labelling}


\paragraph{Low Blank Node Usage}





%%%%%%% Bibliography

% Hogan2012 : Aidan Hogan and Jürgen Umbrich and Andreas Harth and Richard Cyganiak and Axel Polleres and Stefan Decker. An empirical survey of Linked Data conformance. Web Semantics: Science, Services and Agents on the World Wide Web, 2012%


%!TEX root = ../../../diachron-D5_2.tex
%~(required for: \textbf{GEN}, \textbf{DP}, \textbf{EBI})
\subsubsection{Dynamicity Category}
\label{sec:Dynamicity} 

\paragraph{Currency Dimension}~\\ % dimension name 
Virtually all application domains are interested in getting access to data that is as up-to-date as possible. Actually, it is generally the case that data sources decrease in value as they become outdated. Therefore, the currency dimension is a key feature of Linked Data resources. The following metrics, intended to assess the currency of datasets, are based in the definition of this dimension provided by Kahn et al.~\cite{Kahn2002} as ``the degree to which information is up-to-date''.

\paragraph{Currency of Documents/Statements Metric}~(required for: \textbf{GEN}, \textbf{DP}, \textbf{EBI})~\\ %metric name
% background for the metric
This metric is based in the definition of currency provided by Rula et al. as ``the age of a value, where the age of a value is computed as the difference between the current time (the observation time) and the time when the value was last modified''~\cite{Rula2012}. The age of a dataset can be computed at both, the resource level (by comparing its last time of modification with the observation time) and the triples level (by comparing the value of \todo{CL: How would you do this?  If statements are reified?}last-modified statements instead).
% short description
\begin{mdframed}[style=metricdefinition]
\emph{Measures the degree to which data is up to date, by comparing the time when the data was observed (approximately the current time), with the time when the data (the document and each triple) was last modified.}
\end{mdframed}

% pseudocode
As listed in algorithm~\ref{alg:currencyDocsStm}, the metric is computed as the average of the comparisons of the observation time versus the last modification time of each triple, normalized by the time elapsed since the publication of the document and the observation time. The last modification time is extracted from the properties: \textit{http://purl.org/dc/terms/modified} and \textit{http://semantic-mediawiki.org/swivt/1.0\#wikiPageModificationDate}. Likewise, the publication time of the document is extracted from \textit{http://purl.org/dc/terms/created}, \textit{http://purl.org/dc/terms/issued} and \textit{http://semantic-mediawiki.org/swivt/1.0\#creationDate}, as suggested by the study conducted by Rula et al.~\cite{Rula2012}.
\begin{algorithm}
\caption{Currency of Documents/Statements Algorithm} \label{alg:currencyDocsStm}
\begin{algorithmic}[1]
\Procedure{init}{}
\State accumTimeDiffs = 0;
\State countModifiedObjs = 0;
\State observationTime = getCurrentTime();
\State publishingTime = null;
\EndProcedure
\Procedure{compute}{$\langle?s,?p,?o,?g\rangle$}
\If {isLastModifiedTimeURI(?p)} 
\State accumTimeDiffs += (observationTime - parseAsTime(?o));
\State countModifiedObjs++;
\EndIf
\If {isPublishingTimeURI(?p)} 
\State publishingTime = parseAsTime(?o);
\EndIf ~\\
\EndProcedure
\end{algorithmic}
\end{algorithm}
% metric value, range and rating
The metric will return the average of the difference between one and the ratio of the time elapsed since the last modification of each triple to the total time the dataset has been available. Since the former time span cannot be larger that the latter, the expected range is $[0,\ldots,1]$. The higher the value, the better, as it reflects how recently have the resources been updated. A value of 0 indicates that the resources have never been updated after their publication.

% end-of Currency of Documents/Statements Metric

\paragraph{Time Since Modification Metric}~\\ %metric name
% background for the metric
As suggested above, the difference between the observation time (the time at which the resource is examined) and the time when the data was last modified, is a natural measure of how current is the information provided by a dataset. In contrast with the Currency of Documents/Statements metric, a simpler approach can be taken in order to measure the degree to which data is up to date. This metric does so by computing the plain difference between the observation time and the time when the data values were last modified (note that the normalization factor, based on the publishing time of the document, is omitted here).
% short description
\begin{mdframed}[style=metricdefinition]
\emph{Provides a measure of the degree to which information is up to date, by taking the average difference between the observation time (i.e. the instant when the present calculation of the metric was initiated) and the time when the data (each triple in the dataset) were last modified.}
\end{mdframed}

% pseudocode
The computation performed by the metric is detailed in algorithm~\ref{alg:timeSinceMod}. The observation time corresponds to the instant when the calculation of the metric was initiated and only the triples providing a last time of modification are processed during the computation.
\begin{algorithm}
\caption{Time Since Modification Algorithm} \label{alg:timeSinceMod}
\begin{algorithmic}[1]
\Procedure{init}{}
\State accumTimeDiffs = 0;
\State countModifiedObjs = 0;
\State observationTime = getCurrentTime();
\EndProcedure
\Procedure{compute}{$\langle?s,?p,?o,?g\rangle$}
\If {isLastModifiedTimeURI(?p)} 
\State accumTimeDiffs += (observationTime - parseAsTime(?o));
\State countModifiedObjs++;
\EndIf ~\\
\EndProcedure
\end{algorithmic}
\end{algorithm}
% metric value, range and rating
The result is in the range $[0, +\infty]$ and represents an amount of time in milliseconds. The lower the value, the better, as this metric measures the average time elapsed since the last modification of the information contained in the resource. That is, high values indicate that data provided by the resource was last updated long ago.

% end-of Time Since Modification Metric

\paragraph{Exclusion of Outdated Data Metric}~\\ %metric name
% background for the metric
The currency of a Linked Data resource can also be measured relatively to the amount of outdated entities it contains. However, in order to do so, it is necessary to somehow be able recognize when an entity is to be deemed as outdated, which in turn requires that temporal metadata is available and represented according to an appropriate model, such as those proposed by Rula et al. in \cite{Rula2012A}. In this particular case, we consider an entity to be outdated, if information about its period of validity is provided and according to it, data has already expired. The proportion of outdated data present in a dataset is an important quality factor, since it is usually the case that outdated data is no longer valid.
% short description
\begin{mdframed}[style=metricdefinition]
\emph{Determines the extent to which information provided in a dataset is outdated, by comparing the total number of entities described by the dataset, versus how many of those that are recognized to be outdated.}
\end{mdframed}

% pseudocode
The implementation of the metric, as presented in algorithm \ref{alg:exclusionOutdated}, makes use of the property \textit{http://purl.org/dc/terms/valid} (when provided) to determine the expiration time of each examined triple and then, whether the described entity is outdated. After processing all triples of the dataset, the metric is computed as the ratio of the number of outdated entities to the total number of entities in the dataset.
\begin{algorithm}
\caption{Exclusion of Outdated Data Algorithm} \label{alg:exclusionOutdated}
\begin{algorithmic}[1]
\Procedure{init}{}
\State countTotalOutdatedObjs = 0;
\State countTotalDescribedObjs = 0;
\State observationTime = getCurrentTime();
\EndProcedure
\Procedure{compute}{$\langle?s,?p,?o,?g\rangle$}
\If {isValidTimeURI(?p)} 
\If {observationTime $<$ parseAsTime(?o)} 
\State countTotalOutdatedObjs++;
\EndIf
\EndIf
\If {equalsURI(?p, rdf:type) \&\& isURI(?s)} 
\State countTotalDescribedObjs++;
\EndIf ~\\
\EndProcedure
\end{algorithmic}
\end{algorithm}
% metric value, range and rating
The metric values vary in the interval $[0, 1]$, where a value of 1 indicates the best quality (no entities were found to be outdated).
% end-of Exclusion of Outdated Data Metric

\paragraph{Volatility Dimension}~\\ % dimension name 
%\subparagraph{Background for Volatility Quality Metrics} ~\\ %general background
In Linked Data evolution appeared almost at each new published version of data. Following the idea of Papavassiliou et al.~\cite{TODS13} curators could define a list of changes that occur frequently and correspond to one or more low-level changes (added or deleted triples). These changes termed as Simple Changes also in the context of DIACHRON and comprise an upper abstract level of changes which is pilot-specific to describe group of changes that appear a special interest for each pilot. The detection of Simple Changes achieved accordingly to the methodology presented in~\cite{TODS13} and followed in change detection service of DIACHRON (cf.\ Deliverable 3.1~\cite{diachron-d3.1}). The following three volatility metrics take into account these assumptions and background information.

\paragraph{Versions Volatility Metric}~(required for: \textbf{GEN}, \textbf{DP}, \textbf{EBI})~\\ %metric name
% background for the metric
The comparison of two sequential (or not) versions of datasets could contain a number of simple changes for each pilot. In other cases, it makes sense to compare an old version of a dataset with the newest one. 

% short description
\begin{mdframed}[style=metricdefinition]
\emph{Calculates the number of simple changes between two specified versions.}
\end{mdframed}

% pseudocode
The Versions Volatility Metric can be applied to a pair
of defined versions to count the detected number of Simple Changes. This is achieved by querying the corresponding named graph where the the total number of Simple Changes have been stored which are returned as result. 

% algorithm
\begin{algorithm}
\caption{Versions Volatility Algorithm}
\begin{algorithmic}[1]
\Procedure{init}{}
\State numberOfChanges = 0
\EndProcedure
\Procedure{compute}{}
\State numberOfChanges = countSimpleChanges(v1,v2)
\EndProcedure
\end{algorithmic}
\end{algorithm}

% metric value, range and rating
The metric will return the total number of Simple Changes between two versions [integer number].
%%%%%%%%%%%%%%%%%%%%%%%%%%%%%%%%%%%%%%%%%%%%%%%%%%%%%%%%

\paragraph{Average Volatility Metric}~(required for: \textbf{GEN}, \textbf{DP}, \textbf{EBI})~\\ %metric name
% background for the metric
The number of detected simple changes could be varied across different published versions for each curator/pilot. According to different scenarios some versions are similar while others appear many deltas. Thus, it is meaningful to examine all available published versions of datasets in order to find the average detected Simple Changes across each pair of versions. 

% short description
\begin{mdframed}[style=metricdefinition]
\emph{Calculates the average number of simple changes detected across the published versions.}
\end{mdframed}

% pseudocode
The Average Volatility Metric firstly calculates the total number of published versions through a SPARQL query. Afterwards, it calculates the detected Simple Changes per versions pair and aggregates the sum of changes. Finally, it calculates and returns the ratio between aggregated sum and the number of examined pairs.

% algorithm
\begin{algorithm}
\caption{Average Volatility Metric Algorithm}
\begin{algorithmic}[1]

\Procedure{init}{}
\State changesTotal = 0
\State versionsNo = 0
\State retValue = 0
\EndProcedure
\Procedure{compute}{}
\State $ Versions [] = countVersions (SPARQL)$

\ForAll{$v[i],v[i+1] \in Versions$}
\State $changesTotal = changesTotal + countSimpleChanges(v[i],v[i+1])$
\EndFor
\EndProcedure
\State retValue = changesTotal / versionsNo -1
\end{algorithmic}
\end{algorithm}

% metric value, range and rating
The metric will return the ratio $[0,\ldots,1]$ of average detected simple changes across the published dataset versions. 
%%%%%%%%%%%%%%%%%%%%%%%%%%%%%%%%%%%%%%%%%%%%%%%%%%%%%%%%

\paragraph{Weighted Volatility Metric}~(required for: \textbf{GEN}, \textbf{DP}, \textbf{EBI})~\\ %metric name
% background for the metric
In some applications pilots are interested more in evolution of specified versions. By applying a weighted sum model \todo{CL: please add to bib}\cite{WSM} for each sequential pair of versions, we could adapt this preference for each pilot.

% short description
\begin{mdframed}[style=metricdefinition]
\emph{Calculates the average weighted sum of simple changes that has been detected across the published versions.}
\end{mdframed}

% pseudocode
The Weighted Volatility Metric after finding the total number of published versions it loads the weights
from the curator's preference table. Afterwards, it calculates the simple changes per pair and multiply with the corresponding weight. Finally it calculates the ratio of weighted sum of changes to the examined pairs of versions.

% algorithm
\begin{algorithm}
\caption{Weighted Volatility Metric Algorithm}
\begin{algorithmic}[1]

\Procedure{init}{}
\State aggregSChanges = 0
\State versionsNo = 0
\State retValue = 0
\EndProcedure

\Procedure{LoadWeights}{}
\State $ weights [] = fetchWeights()$
\EndProcedure

\Procedure{compute}{}
\State $ versions [] = countVersions (SPARQL)$

\ForAll{$v[i],v[i+1] \in versions$}
	\ForAll{$w[j] \in weights$}
\State $changesTotal = changesTotal + w[j]*countSimpleChanges(v[i],v[i+1])$
	\EndFor
\EndFor
\EndProcedure
\State retValue = aggregSChanges / versionsNo -1
\end{algorithmic}
\end{algorithm}

% metric value, range and rating
The metric will return the ratio of $[0,\ldots,1]$ aggregated weighted sum detected simple changes across the published dataset versions.

\paragraph{Time Validity Interval Metric}~(required for: \textbf{GEN}, \textbf{DP}, \textbf{EBI})~\\ %metric name
% background for the metric
An alternative to measure how frequently the data varies with time, is to make use of information about the expiration of entities in order to assess how long does data remain valid. Consequently, such measurement would require two pieces of information about entities described in the dataset: the expiry time of entities (i.e. validity) and the time when data became available (i.e. publishing time). By combining these two components, the interval during which the information will remain valid can be computed.
% short description
\begin{mdframed}[style=metricdefinition]
\emph{Estimates the frequency with which data will be updated, by calculating the average length of the time interval during which entities remain valid.}
\end{mdframed}

% pseudocode
As shown in algorithm \ref{alg:timeValidityInterval}, information about the expiration time and issued time of entities is extracted from the triples, thereby accumulating the data necessary to compute the average length of the validity interval of all entities for which such information is provided in the dataset.
\begin{algorithm}
\caption{Time Validity Interval Algorithm} \label{alg:timeValidityInterval}
\begin{algorithmic}[1]
\Procedure{init}{}
\State mapObjsWithValidityInfo = new Map$\langle URI, Object\rangle$();
\EndProcedure
\Procedure{compute}{$\langle?s,?p,?o,?g\rangle$}
\If {isValidityTimeURI(?p)} 
\State setValidityTimeOfObj(getElementWithKey(mapObjsWithValidityInfo, ?s), parseAsTime(?o));
\EndIf
\If {isPublishingTimeURI(?p)} 
\State setIssueTimeOfObj(getElementWithKey(mapObjsWithValidityInfo, ?s), parseAsTime(?o));
\EndIf ~\\
\EndProcedure
\end{algorithmic}
\end{algorithm}
% metric value, range and rating
The result is in the range $[0, +\infty]$ and represents an amount of time in seconds. Lower values correspond to better quality rankings, as they indicate that data is expected to be updated more frequently and therefore would be fresher.

% end-of Time Validity Interval Metric

\paragraph{Timeliness Dimension}~\\ % dimension name 
The degree to which Linked Data is considered to be up-to-date or outdated can heavily depend on the task at hand. Depending on the application, information updated one month ago can be considered highly current (e.g. the list of pet shops in a city) or unacceptably outdated (e.g. a pricing list of foreign currencies). Therefore, in order to align the quality measures related to the age of data with the application of interest, it makes sense to provide currency measures relative to a specific task. The metrics for this dimension, are based on the concept of timeliness as defined by Gamble et al.: ``Timeliness is a measure of utility, is a comparison of the date the annotation was updated with the consumer's requirement''~\cite{Gamble2011}.

\paragraph{Timeliness of the Resource Metric}~(required for: \textbf{GEN}, \textbf{DP}, \textbf{EBI})~\\ %metric name
% background for the metric
If the expiration time of the data comprised by a Linked Data resource is provided, it is reasonable to expect such data to be updated before or shortly after that time. Otherwise, it would suggest that the data might be outdated. This, added up to the fact that the expiration time of information is closely associated to the application domain, allows to define Timeliness of the Resource as an application-domain-related metric that measures the currency of data.
% short description
\begin{mdframed}[style=metricdefinition]
\emph{Indicates how up-to-date data is, relative to a specific task, by measuring the difference between the invalid time (expiry time of the data) and the observation time (current time).}
\end{mdframed}

% pseudocode
Algorithm \ref{alg:timelinessResource} illustrates how this metric is computed. The procedure determines whether the triple contains  temporal information stating when the described entity expires (i.e. its validity). This is done by evaluating the property regarded as source  of the Expiration/Valid Time (namely, \textit{http://purl.org/dc/terms/valid}, as suggested by Rula et al. \cite{Rula2012}), if such a property is found, its value is subtracted from the current observation time, and the result is accumulated. The total accumulated differences will be used afterwards, to calculate the final value of the metric.
\begin{algorithm}
\caption{Timeliness of the Resource Algorithm} \label{alg:timelinessResource}
\begin{algorithmic}[1]
\Procedure{init}{}
\State accumValidTimeDiffs = 0;
\State countTotalAccountedObjs = 0;
\State observationTime = getCurrentTime();
\EndProcedure
\Procedure{compute}{$\langle?s,?p,?o,?g\rangle$}
\If {isExpirationTimeURI(?p)} 
\State accumValidTimeDiffs += (observationTime - parseAsTime(?o));
\State countTotalAccountedObjs++;
\EndIf
\EndProcedure
\end{algorithmic}
\end{algorithm}
% metric value, range and rating
The result of the metric is a real number in the range $[-\infty, +\infty]$, as it represents the average length of the gap (in  milliseconds) between the expiration time of the data and the actual time. The lower the value, the better, since a higher, positive value indicates that the resources are possibly outdated.

% end-of Timeliness of the Resource Metric

%%% Local Variables: 
%%% mode: latex
%%% TeX-master: "../../../diachron-D5_2"
%%% End: 

%!TEX root = ../../../diachron-D5_2.tex

\subsubsection{Contextual Category}
\label{sec:Contextual} 


%..add more categories
