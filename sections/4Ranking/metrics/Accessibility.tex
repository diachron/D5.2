%!TEX root = ../../../diachron-D5_2.tex

\subsubsection{Accessibility Category}
\label{sec:Accessibility} 

\paragraph{Availability Dimension} % dimension name 

\paragraph{Dereferenceability Metric} ~\\ %metric name
% background for the metric
HTTP URIs should be dereferencable, i.e. HTTP clients can retrieve the resources identified by the URI.
A typical web URI resource would return a \texttt{200 OK} code indicating that a request is successful and \texttt{4xx} or \texttt{5xx} if the request is unsuccessful. 
In Linked Data, a successful request should return a document (RDF) containing the description (triples) of the requested resource.
In Linked Data, there are two possible ways which allow publishers make URIs dereferencable.
These are the \texttt{303} URIs and the \texttt{hash} URIs\footnote{http://www.w3.org/TR/cooluris/}.
Yang et. al~\cite{Yang2011} describes a mechanism to identify the dereferenceability process of linked data resource.
% short description
\begin{mdframed}[style=metricdefinition]
\emph{Calculates the number of valid redirects (303) or hashed links according to LOD Principles.}
\end{mdframed}

% pseudocode
This metric (listing~\ref{let:deref}) will count the number of valid dereferenceable URI resources found in the subject (?s) and object (?o) position of a triple. The \texttt{isDereferenceable(resource)} method uses the rules defined in~\cite{Yang2011}.
\begin{algorithm}
\caption{Dereferenceablity Algorithm}\label{lst:deref}
\begin{algorithmic}[1]
\Procedure{init}{}
\State totalTriples = 0 ;
\State deref = 0 ;
\EndProcedure

\Procedure{Dereference}{$\langle?s,?p,?o,?g\rangle$}
\If {(isURI(?s)) \&\& (isDereferenceable(?s))} deref++ ; \EndIf

\If {(isURI(?o)) \&\& (isDereferenceable(?o))} deref++ ; \EndIf
\State totalTriples++;
\EndProcedure
\end{algorithmic}
\end{algorithm}
% metric value, range and rating
The metric will return a ratio of the number of dereferenced URIs (deref) against the total number of triples in a dataset (totalTriples). The expected range is [0..1], where 0 is the worst rating and 1 is the best rating.


%%%%%%% Bibliography
% Yang2011 : https://dl.dropboxusercontent.com/u/4138729/paper/dereference_iswc2011.pdf