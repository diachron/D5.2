%!TEX root = ../../../diachron-D5_2.tex
%~(required for: \textbf{GEN}, \textbf{DP}, \textbf{EBI})
\subsubsection{Contextual Category}
\label{sec:Contextual} 

\paragraph{Amount of Data Dimension}~\\ % dimension name 
Amount of data, similar to relevancy, is all about the volume of appropriate data for the task at hand.
Flemming defined this dimension as the ``criterion influencing the usability of a data source''~\cite{Flemming2008}
The only metric available in this dimension is purely statistical.

\paragraph{Amount of Triples Metric}~(required for: \textbf{GEN})~\\ %metric name
% background for the metric
This metric counts the number of triples present in the dataset. 
It will check on the size of the dataset which is the lower bound for the number of triples present in the dataset.
In order to identify the volume of the data a range\footnote{Taken from \url{http://lod-cloud.net}} was defined as follows:
\begin{itemize}
\item high (metric value 1)\:  > 1000000000 triples;
\item mediumHigh (metric value 0.8)\: 10000000 triples~$<~x~\leq~$ 1000000000 triples;
\item medium (metric value 0.6)\: 500000 triples~$<~x~\leq~$ 10000000 triples;
\item mediumLow (metric value 0.4)\: 10000 triples~$<~x~\leq~$ 500000 triples;
\item low (metric value 0.2)\: < 10000 triples.
\end{itemize}

% short description
\begin{mdframed}[style=metricdefinition]
\emph{Measures the size of the dataset in terms of the number of triples.}
\end{mdframed}

The algorithm counts the number of triples and returns a value according to the range defined above.