%!TEX root = ../../../diachron-D5_2.tex

\subsubsection{Intrinsic Category}
\label{sec:Intrinsic} 


\subsubsection{Accuracy} % dimension name 

\paragraph{Malformed Datatype Literals} ~\\ %metric name
% background for the metric
% short description
\begin{mdframed}[style=metricdefinition]
\emph{Calculates the ratio of typed literals valid regarding its given xsd datatype to all literals}
\end{mdframed}

% pseudocode

This metric (listing~\ref{let:deref}) will count the number of valid dereferenceable URI resources found in the subject (?s) and object (?o) position of a triple. The \texttt{isDereferenceable(resource)} method uses the rules defined in~\cite{Yang2011}.
\begin{algorithm}
\caption{Dereferenceablity Algorithm}\label{lst:deref}
\begin{algorithmic}[1]
\Procedure{init}{}
\State totalTriples = 0 ;
\State deref = 0 ;
\EndProcedure

\Procedure{Dereference}{$\langle?s,?p,?o,?g\rangle$}
\If {(isURI(?s)) \&\& (isDereferenceable(?s))} deref++ ; \EndIf

\If {(isURI(?o)) \&\& (isDereferenceable(?o))} deref++ ; \EndIf
\State totalTriples++;
\EndProcedure
\end{algorithmic}
\end{algorithm}
% metric value, range and rating

\paragraph{Literals Incompartible with Datatype Range} 
\paragraph{Defined Ontology Author}
\paragraph{POBO Definition Usage}
\paragraph{Synonym Usage}
\subsubsection{Consistency}
\

\paragraph{Entities As Members of Disjoint Classes}
\paragraph{Homogeneous Datatypes}
\paragraph{Misplaced Classes or Properties}
\paragraph{Misused Owl Datatype or Object Properties}
\paragraph{Obsolete Concepts in Ontology}
\paragraph{Ontology Hijacking}
\paragraph{Undefined Classes or Properties}

\subsubsection{Conciseness}

\paragraph{Duplicate Instance}

\paragraph{Extensional Conciseness}

\paragraph{Ontology Versioning Conciseness}



