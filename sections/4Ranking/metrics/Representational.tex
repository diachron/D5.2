%!TEX root = ../../../diachron-D5_2.tex

\subsubsection{Representational Category}
\label{sec:Representational} 


\subsubsection{Representational Conciseness}

Metrics in the representational dimensions capture aspects related to information representation.


\paragraph{Short URIs}

\subsubsection{Understandability}


\paragraph{Empty Annotation Value}

Some languages, e.g. OWL distinguish annotation properties.
Annotation properties are predicates that provide informal documentation annotations about ontologies, statements, or IRIs. 
A simple example for annotation property is \textit{rdfs:comment} which  is used to provide a comment. 
Unfortunately annotation properties are often used with empty literal values that couse inconsistences in data.
The problem can be solved by the corresponding triples or by replacing empty literals by annotation strings.
The following annotation properties were used in this metric:
\begin{itemize}
\item \textit{skos:altLabel}
\item \textit{skos:hiddenLabel}
\item \textit{skos:prefLabel}
\item \textit{skos:changeNote}
\item \textit{skos:definition}
\item \textit{skos:editorialNote}
\item \textit{skos:example}
\item \textit{skos:historyNote}
\item \textit{skos:note}
\item \textit{skos:scopeNote}
\item \textit{dcterms:description}
\item \textit{dc:description}
\item \textit{rdf:label}
\item \textit{rdf:comment}
\end{itemize}

% background for the metric
The metric Empty Annotation Value identifies triples whose property is an annotation property and whose object is an empty string.
 
% short description
\begin{mdframed}[style=metricdefinition]
\emph{Calculates the ratio of annotations with empty values to all annotations in the data set.}
\end{mdframed}

% pseudocode
\begin{algorithm}
\caption{Empty Annotation Value Algorithm}\label{lst:emptyAnnotationValue}
\begin{algorithmic}[1]
\Procedure{init}{}
\State totalAnnotations = 0 ;
\State emptyAnnotations = 0 ;
\EndProcedure

\Procedure{CountEmptyAnnotations}{$\langle?s,?p,?o,?g\rangle$}
\If {(isAnnotation(?p))}
\If {(isEmpty(?o))} emptyAnnotations++ ; \EndIf
\State totalAnnotations++;
\EndIf
\EndProcedure
\end{algorithmic}
\end{algorithm}


\paragraph{Human Readable Labelling}
\paragraph{Labels Using Capitals}
\paragraph{Low Blank Node Usage}
\paragraph{Whitespace in Annotation}